\documentclass[12pt,letterpaper]{article}
\usepackage[utf8]{inputenc}
\usepackage[francais]{babel}
\usepackage[T1]{fontenc}
\usepackage{amsmath}
\usepackage{amsfonts}
\usepackage{amssymb}
\usepackage{amsthm}
\usepackage{amsbsy}
\usepackage{color}
\usepackage[left=1.5cm,right=1.5cm,top=1.5cm,bottom=1.5cm]{geometry}
\usepackage{dsfont}
%Permet d'afficher la page de référence dans la ToC

% URL
\usepackage{url}

\usepackage{hyperref}

\usepackage{multicol}

% Modification du format des numéros de pages.
\usepackage{fancyhdr}
\usepackage{lastpage}
%-----------------------------------------------------------------
%-----------------------------------------------------------------
% INFORMATIONS À REMPLIR

\newcommand{\TypeRemise}{Proposition de Projet}
\newcommand{\sigleCours}{IFT6390}
\newcommand{\titreCours}{Fondements de l'apprentissage machine}

% Coéquipier UN
\newcommand{\PrenomUn}{Gabriel}
\newcommand{\NomUn}{Lemyre}

% Coéquipier DEUX
\newcommand{\PrenomDeux}{Guillaume}
\newcommand{\NomDeux}{Poirier-Morency}

\newcommand{\PrenomProf}{Aaron}
\newcommand{\NomFamilleProf}{Courville}

\newcommand{\dateremise}{14 novembre 2017}

%-----------------------------------------------------------------

\newcommand{\CreateurNom}[2]{#1 \textsc{#2}\ }
\newcommand{\CoUn}{\CreateurNom{\PrenomUn}{\NomUn}}
\newcommand{\CoDeux}{\CreateurNom{\PrenomDeux}{\NomDeux}}
\newcommand{\NomProf}{\CreateurNom{\PrenomProf}{\NomFamilleProf}}

%-----------------------------------------------------------------
% Changement du style de mise en page
	\pagestyle{fancy}
	\fancyhf{}
 	\fancyhead[L]{\sigleCours \\ - \TypeRemise}
 	\fancyhead[C]{--- \CoUn --- \\ --- \CoDeux ---}
 	\fancyhead[R]{Remise \\ \dateremise}

 %-----------------------------------------------------------------
 % Modifier style du numéro de page
\rfoot{\vspace*{-2em} Page \thepage}


\begin{document}

\textbf{Titre:} Thomas Bayes aurait-il dû croquer le fruit interdit?

\abstract

L'objectif de ce projet est d'expérimenter différentes techniques de
pré-traitement et d'évaluer la performance de généralisation qu'elles
permettent pour une variété de modèles de classification. Les même modèles
seront appliqués sur deux jeux de données fondamentalement différent et
permetteront de déterminer quel modèle s'applique mieux dans quel cas.
\\

En particulier, nous voulons expérimenter avec un arbre de décision hybride
visant à minimiser la covariance des composantes afin de pouvoir utiliser un
classifieur Bayésien naïf. Nous comparerons cette approche à la décomposition
en composantes primaires. Nous étudierons aussi l'utilisation de distributions
mixtes pour construire des noyaux. Cette approche est justifiée par le fait que
le jeu de données de prédiction du salaire présente une combinaison d'attributs
catégoriques et qu'un arbre de décision nous permettrait de nous en affranchir.
\\

En deuxième temps, nous expérimenterons avec les réseaux de neurones
convolutifs et les auto-encodeurs afin d'étudier l'échantillon MNIST. Puisque
les applications de ces modèles de reconnaissance de motifs sont quasi
illimitée, nous pourrons maîtriser des outils capable de traiter des problèmes
plus complexes comme la reconnaissance de motifs de chaleur précurseurs aux
feux de forêt sur de l'imagerie satellite.

\section{Pré-traitements}

En plus d'appliquer les algorithmes qui suivent sur nos données, nous allons
également vérifier l'impact des pré-traitements suivants:

\begin{itemize}
\item décomposition en composante primaires (PCA)
\item arbre de décision minimisant la covariance de sorte que l'hypothèse
      d'indépendance soit plus justifiée pour l'utilisation d'un Bayes naïf.
\item auto-encodeur
\item réseau convolutif
\end{itemize}

Nous allons également explorer la possiblité d'utiliser des auto-encoders pour
traiter l'image. Une approche serait d'associer aux éléments d'une classe une
permutation aléatoire d'exemplaires à reproduire de sorte que l'encodeur ne
puisse se contenter d'apprendre la fonction identité, mais plutôt une
représentation intermédiaire potentiellement utile.

Nous avons choisi le modèle de réseau de neurone convolutif d'une part pour sa
popularité et aussi pour mettre en pratique et explorer des outils comme
PyTorch. Le modèle de réseau convolutif consiste à combiner l'information
locale d'une image en appliquant le même réseau de neurone sur différentes
régions et de mettre en commun les résultats pour construire une représentation
plus compacte. L'avantage de ce type de réseau est qu'il est invariant aux
translations des caractéristiques de l'image d'entrée et qu'il nécessite peu de
pré-traitement puisqu'il construit lui-même une bonne représentation de l'image.
\\

Le pré-traitement utilisé dépendra du jeu de données.

\section{Algorithmes}

Le classifieur de Bayes nous intéresse beaucoup, car il offre un bon cadre pour
expérimenter divers modèle de densité à priori et par conséquent d'astuce
de noyau. De plus, la variante naïve est une bonne base de référence pour
comparer les modèle d'apprentissage statistique.

Le modèle de Bayes se base en deux temps: on estime une densité à priori sur la
classe $P[y^i = c_j]$ et ensuite on ajoute l'information sur l'exemplaire
$P[y^i = c_j | x^i]$. On utilise la règle de Bayes pour calculer la probabilité
à postériori. Finalement, on choisit la classe $c_j$ associée à la plus grande
probabilité.
\\

Un arbre de décision est un algorithme permettant de classifier un exemplaire
en suivant des arcs étiquetés par des prédicats sur ses attributs. Chaque
feuille est étiquetée par une classe. La topologie et les attributs sont
choisis en fonction d'hyper-paramètres et d'une métrique au moment de
l'entraînement. Par exemple, on pourrait chercher à maximiser le gain
d'information (i.e.  différence d'entropie) résultant du choix d'un attribut.
\\

Le réseau de neurone multi-couche (i.e. perceptron multi-couche) est un modèle
inspiré des connexions entre neurones et synapses dans le cerveau. Il est
composé d'arcs pondérés et de noeuds possédant une fonction d'activation
produisant une sortie sur une combinaison linéaire des activations d'entrées
pondéré par les poids des arcs. Nous allons expérimenter les différentes
architectures pour adresser le problème de classification de salaire et
d'image.
\\

Les algorithmes seront appliqués sur chaque jeu de données et seront combinés
avec des pré-traitements spécifiques.

\section{Bases de données}
\subsection{\href{http://archive.ics.uci.edu/ml/datasets/Adult}{Prédiction du salaire}}

Cette base de données est destinée à prédire si une personne gagne plus
de $50 000$\$ annuellement. Cette base de données est constituée de 14 attributs
(6 dont la distribution est continue et 8 dont les valeurs sont catégorielles)
et d'une sortie $y \in \left( >50K, <=50K \right)$. Elle contient $48 842$
observations, dont certaines sont manquantes, recueillies aux États-Unis en
1994. Cette base de données a l'avantage d'être déjà structurée et d'avoir des
attributs bien définis.

% --------------------------------------------------------------------------------
\subsection{\href{http://yann.lecun.com/exdb/mnist/}{MNIST}}

MNIST est une base de donnée d'image de numéros décimaux écrits à la main
contenant 60000 exemplaires d'entraînement et 10000 exemplaires de test. Elle a
été très populaire au début des années 2000 et est encore utilisée utilisé de
nos jours pour évaluer la performance de généralisation de réseaux de neurones.

\end{document}
