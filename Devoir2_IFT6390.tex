\documentclass[11pt,letterpaper]{article}
\usepackage[utf8]{inputenc}
\usepackage[francais]{babel}
\usepackage[T1]{fontenc}
\usepackage{amsmath}
\usepackage{amsfonts}
\usepackage{amssymb}
\usepackage{amsthm}
\usepackage{amsbsy}
\usepackage{color}
\usepackage[left=1.5cm,right=1.5cm,top=1.5cm,bottom=1.5cm]{geometry}
\usepackage{dsfont}
%Permet d'afficher la page de référence dans la ToC

% URL
\usepackage{url}

% Permet de cross-out un morceau d'une equation dans textmode et mathmode sauf \cancelto{\infty}{} qui fonctionne juste en mathmode
\usepackage[makeroom]{cancel}

\usepackage{multicol}

% Modification du format des numéros de pages.
\usepackage{fancyhdr}
\usepackage{lastpage}
%-----------------------------------------------------------------
%-----------------------------------------------------------------
% INFORMATIONS À REMPLIR

\newcommand{\TypeRemise}{Devoir 2}
\newcommand{\sigleCours}{IFT6390}
\newcommand{\titreCours}{Fondements de l'apprentissage machine}

% Coéquipier UN
\newcommand{\PrenomUn}{Gabriel}
\newcommand{\NomUn}{Lemyre}

% Coéquipier DEUX
\newcommand{\PrenomDeux}{Guillaume}
\newcommand{\NomDeux}{Poirier-Morency}

\newcommand{\PrenomProf}{Aaron}
\newcommand{\NomFamilleProf}{Courville}

\newcommand{\dateremise}{30 novembre 2017}

%-----------------------------------------------------------------

\newcommand{\CreateurNom}[2]{#1 \textsc{#2}\ }
\newcommand{\CoUn}{\CreateurNom{\PrenomUn}{\NomUn}}
\newcommand{\CoDeux}{\CreateurNom{\PrenomDeux}{\NomDeux}}
\newcommand{\NomProf}{\CreateurNom{\PrenomProf}{\NomFamilleProf}}

%-----------------------------------------------------------------
% Changement du style de mise en page
	\pagestyle{fancy}
	\fancyhf{}
 	\fancyhead[L]{\sigleCours \\ - \TypeRemise}
 	\fancyhead[C]{--- \CoUn --- \\ --- \CoDeux ---}
 	\fancyhead[R]{\dateremise}
 	
 %-----------------------------------------------------------------
 % Modifier style du numéro de page
\rfoot{\vspace*{-2em} Page \thepage}
%-----------------------------------------------------------------
% Permet d'ajouter entre autre, une ligne à gauche des exemples,théorèmes,etc...


\usepackage[framemethod=TikZ]{mdframed}
\newcounter{Question}

\newenvironment{Question}[1][]{%
	\refstepcounter{Question}% increment the environment's counter
	\vspace{100pt}
	\begin{mdframed}[%
		frametitle={{\huge{\theQuestion}} #1},
		%frametitlebackgroundcolor=gray!30,
	    topline=true,
	    frametitlerule=true,
	    bottomline=true,
	    rightline=false,
	    leftline=true,
	    nobreak=false,
	    innerbottommargin=1em,
	    %innerleftmargin=1.3em,
		%frametitleaboveskip=10pt, frametitlebelowskip=10pt,
		font=\Large
	]%
}{%
    \end{mdframed}
}


\newcounter{subQcounter}
\newenvironment{SousQuestion}[2][]{%
	\refstepcounter{subQcounter}
	~\\
	~\\
	\fbox{\textbf{\theQuestion.\thesubQcounter}
	\begin{minipage}{\textwidth}
	\textbf{#1}
	\end{minipage}}
	#2
}

\newenvironment{Cas}[1][]{%
	\begin{mdframed}[%
		frametitle={#1},
		frametitlebackgroundcolor=gray!30,
	    topline=true,
	    frametitlerule=true,
	    bottomline=true,
	    rightline=true,
	    leftline=true,
	    nobreak=false,
	    %innerleftmargin=1.3em,
		%frametitleaboveskip=10pt, frametitlebelowskip=10pt,
		font=\Large
	]%
}{%
    \end{mdframed}
}

%-----------------------------------------------------------------
%------------------------------
%Environnements
%------------------------------

%Format des théorèmes/definitions et exemples
\theoremstyle{definition}

%--Format de nouvel environnement
\usepackage{thmtools}
%\newtheoremstyle{<name>}%
 % spaceabove={<space above>}%
 % spacebelow={<space below>}%
 % {<body font>}%
 % {<indent amount>}%
 % headfont={<theorem head font>}%
 % {<punctuation after theorem head>}%
 % {<space after theorem>}%

%Format pour des adresse web
\usepackage{url}

%
\usepackage{upgreek}

%Tableaux et figures
\usepackage{tabularx}
\usepackage{graphicx}
\usepackage{caption}
\usepackage{booktabs}
\usepackage{wrapfig}

%Placer des ancres dans le texte et y faire référence
\usepackage[pageanchor]{hyperref}
%Permet que les numéros de pages de la ToC soit des hyperliens
%\usepackage[linktocpage=true]{hyperref}
\usepackage{tocloft}
\renewcommand{\cftparapresnum}{\S}



%Pour prendre du code R et l'inclure en format LaTeX
\usepackage{listings}

%-----------------------------------------------------------------
\definecolor{dkgreen}{rgb}{0,0.4,0}
\definecolor{gray}{rgb}{0.5,0.5,0.5}
\definecolor{mauve}{rgb}{0.58,0,0.82}

% Inclure code R dans le texte -------------------------
\lstset{ %
  language=R,                     % the language of the code
  basicstyle=\footnotesize,       % the size of the fonts that are used for the code
  numbers=left,                   % where to put the line-numbers
  numberstyle=\tiny\color{gray},  % the style that is used for the line-numbers
  stepnumber=1,                   % the step between two line-numbers. If it's 1, each line
                                  % will be numbered
  numbersep=5pt,                  % how far the line-numbers are from the code
  backgroundcolor=\color{white},  % choose the background color. You must add \usepackage{color}
  showspaces=false,               % show spaces adding particular underscores
  showstringspaces=false,         % underline spaces within strings
  showtabs=false,                 % show tabs within strings adding particular underscores
  frame=single,                   % adds a frame around the code
  rulecolor=\color{black},        % if not set, the frame-color may be changed on line-breaks within not-black text (e.g. commens (green here))
  tabsize=2,                      % sets default tabsize to 2 spaces
  captionpos=b,                   % sets the caption-position to bottom
  breaklines=true,                % sets automatic line breaking
  breakatwhitespace=false,        % sets if automatic breaks should only happen at whitespace
  title=\lstname,                 % show the filename of files included with \lstinputlisting;
                                  % also try caption instead of title
  keywordstyle=\color{blue},      % keyword style
  commentstyle=\color{dkgreen},   % comment style
  stringstyle=\color{dkgreen},      % string literal style
  escapeinside={\%*}{*)},         % if you want to add a comment within your code
  morekeywords={*,...}            % if you want to add more keywords to the set
} 
%------------------------------
%espacement
%------------------------------
\usepackage{setspace}
%Puis changer les option d'espacement:
%\doublespacing
%\singlespacing
%\onehalfspacing
%\setstretch{1.2}

%operateur e
\newcommand{\e}[1]{e^{#1}}


%\newcolumntype{C}{ >{\centering\arraybackslash} m{2cm} }
%\newcolumntype{d}{ >{\centering\arraybackslash} m{0.7cm} }
\usepackage{xfrac}

\usepackage{colortbl}
\definecolor{silver}{RGB}{220,220,220}

\usepackage{epstopdf}
\epstopdfDeclareGraphicsRule{.tif}{png}{.png}{convert #1 \OutputFile}
\AppendGraphicsExtensions{.tif}



% Espérance 
\long\def\esp#1{\mathds{E}\!\left[ #1 \right]}
% Exponentielle
\def\ex#1{e^{#1}}
% Fonction Indicatrice
\def\ind#1{\mathbf{1}\! \text{l}_{ \{ #1 \} }}
% def au dessus du signe d'égalité
\newcommand\myeq{\stackrel{\mathclap{\normalfont\mbox{def}}}{=}}
%somme avec limite inf et sup et le contenu comme dernier argument
\def\somme#1#2#3{\sum\limits_{#1}^{#2} {#3}}

% FONT FOR R FUNCTION
\newcommand{\rfct}[1]{\texttt{#1}} 

% VALEURS ABSOLUES ET NORMES
\usepackage{mathtools}
\DeclarePairedDelimiter\abs{\lvert}{\rvert}%
\DeclarePairedDelimiter\norm{\lVert}{\rVert}%

% Swap the definition of \abs* and \norm*, so that \abs
% and \norm resizes the size of the brackets, and the 
% starred version does not.
\makeatletter
\let\oldabs\abs
\def\abs{\@ifstar{\oldabs}{\oldabs*}}
%
\let\oldnorm\norm
\def\norm{\@ifstar{\oldnorm}{\oldnorm*}}
\makeatother

\def\xt{$\{X_t\}$\ }
\def\xtm{$X^{t}_{t+m}$}
\def\sigw{\sigma_{\omega}^2}

\newcommand*{\ovA}[1]{%
  \overline{\mbox{#1}\raisebox{4.5mm}{}}
}

\def\Rcon{\ovA{$R$}}

% Créer une fonction qui entoure d'un cercle la string
\usepackage{mathdesign}

\renewcommand\qedsymbol{$\blacksquare$}
% Changer l'environnement de preuve
\expandafter\let\expandafter\oldproof\csname\string\proof\endcsname
\let\oldendproof\endproof
\renewenvironment{proof}[1][\proofname]{%
  \oldproof[\underline{\textbf{#1}}]%
}{\oldendproof}

\begin{document}

\begin{titlepage}
	\centering
	%\includegraphics[width=0.15\textwidth]{example-image-1x1}\par\vspace{1cm}
	{\scshape\LARGE Université de Montréal \par}
	\vspace{1cm}
	{\scshape\Large \TypeRemise \par}
	\vspace{1.5cm}
	{\huge\bfseries \sigleCours \ - \textsc{\titreCours} \par}
	\vspace{2cm}
	{\Large\itshape \CoUn \\ et \\ \CoDeux \par}
	\vfill
	Remis à \par
	{\Large\itshape \NomProf \\}

	\vfill

% Bottom of the page
	{\large \dateremise \par}
\end{titlepage}
\clearpage

%-----------------------------------------------------------------
%-----------------------------------------------------------------

\begin{Question}[PARTIE THÉORIQUE A (20 pts) : relations et dérivées de quelques fonctions de base]

\begin{SousQuestion} % 1.1
\begin{proof}{Montrez que $sigmoid(x) = \frac{1}{2} \left[ tanh(\frac{1}{2} x) + 1 \right]$}
\begin{align*}
\frac{1}{2} \left[ tanh\left(\frac{1}{2} x\right) + 1 \right] & = \frac{1}{2} \left[ \frac{\ex{\frac{1}{2}x}-\ex{-\frac{1}{2}x}}{\ex{\frac{1}{2}x}+\ex{-\frac{1}{2}x}} + 1 \right] \\
& = \frac{1}{2} \left[ \frac{\ex{\frac{1}{2}x}\cancel{-\ex{-\frac{1}{2}x}} + \ex{\frac{1}{2}x}
\cancel{+\ex{-\frac{1}{2}x}}}{\ex{\frac{1}{2}x}+\ex{-\frac{1}{2}x}} \right] \\
& = \frac{1}{2} \left[ \frac{2\ex{\frac{1}{2}x}}{\ex{\frac{1}{2}x}+\ex{-\frac{1}{2}x}} \right] \\
& = \frac{\ex{\frac{1}{2}x}}{\ex{\frac{1}{2}x}+\ex{-\frac{1}{2}x}} \\
& = \frac{1}{1+\ex{-\frac{1}{2}x-\frac{1}{2}x}} \\
& = \frac{1}{1+\ex{-x}}\\
& = sigmoid(x)
\end{align*}
\end{proof}
\end{SousQuestion}


\begin{SousQuestion} % 1.2
\begin{proof}{Montrez que $ln\left(sigmoid(x)\right) = - softplus(-x)$}
\begin{align*}
ln\left(sigmoid(x)\right) & = ln\left(\frac{1}{1+\ex{-x}}\right) \\
& = \cancelto{0}{ln\left( 1 \right)} - ln\left(1+\ex{-x}\right) \\
& = - ln\left(1+\ex{-x}\right) \\
& = - softplus(-x)
\end{align*}
\end{proof}
\end{SousQuestion}


\begin{SousQuestion} % 1.3
\begin{proof}{Montrez que $\frac{d}{dx}sigmoid(x) = sigmoid(x)\left(1-sigmoid(x)\right)$}
\begin{align*}
\frac{d}{dx}sigmoid(x) & = \frac{d}{dx}\left(\frac{1}{1+\ex{-x}}\right) \\
& = \frac{\cancelto{0}{\frac{d}{dx}(1)} \left(1+\ex{-x}\right) - \frac{d}{dx}\left(1+\ex{-x}\right)}{\left(1+\ex{-x}\right)^2} \\
& = \frac{- \frac{d}{dx}\left(1+\ex{-x}\right)}{\left(1+\ex{-x}\right)^2} \\
& = \frac{\ex{-x}}{\left(1+\ex{-x}\right)^2} \\ \\
1-sigmoid(x) & = 1-\frac{1}{1+\ex{-x}} \\
& = \frac{1+\ex{-x}-1}{\left(1+\ex{-x}\right)} \\
& = \frac{\ex{-x}}{\left(1+\ex{-x}\right)} \\ \\
sigmoid(x)\left(1-sigmoid(x)\right) & = \left[\frac{1}{\left(1+\ex{-x}\right)}\right]\left[\frac{\ex{-x}}{\left(1+\ex{-x}\right)}\right] \\
& = \frac{\ex{-x}}{\left(1+\ex{-x}\right)^2} \\
& = \frac{d}{dx}sigmoid(x)
\end{align*}
\end{proof}
\end{SousQuestion}


\begin{SousQuestion} % 1.4
\begin{proof}{Montrez que $\frac{d}{dx}tanh(x) = 1-tanh(x)^2$}
\begin{align*}
\frac{d}{dx}tanh(x) & = \frac{d}{dx}\left(\frac{\ex{x}-\ex{-x}}{\ex{x}+\ex{-x}}\right) \\
& = \frac{\left[\frac{d}{dx}\left(\ex{x}-\ex{-x}\right)\right]\left(\ex{x}+\ex{-x}\right)-\left(\ex{x}-\ex{-x}\right)\left[\frac{d}{dx}\left(\ex{x}+\ex{-x}\right)\right]}{\left(\ex{x}+\ex{-x}\right)^2} \\
& =\frac{\left(\ex{x}+\ex{-x}\right)\left(\ex{x}+\ex{-x}\right)-\left(\ex{x}-\ex{-x}\right)\left(\ex{x}-\ex{-x}\right)}{\left(\ex{x}+\ex{-x}\right)^2} \\
& = \frac{\left(\ex{x}+\ex{-x}\right)^2 - \left(\ex{x}-\ex{-x}\right)^2}{\left(\ex{x}+\ex{-x}\right)^2} \\ \\
1-tanh(x)^2 & = 1 - \left(\frac{\ex{x}-\ex{-x}}{\ex{x}+\ex{-x}}\right)^2 \\
& = \frac{\left(\ex{x}+\ex{-x}\right)^2 - \left(\ex{x}-\ex{-x}\right)^2}{\left(\ex{x}+\ex{-x}\right)^2} \\
& = \frac{d}{dx}tanh(x)
\end{align*}
\end{proof}
\end{SousQuestion}


\begin{SousQuestion} % 1.5
\par Puisque $sign(x)$ est défini comme 
\begin{align*}
sign(x)=\begin{cases}
+1 & x>0, \\
0  & x=0, \\
-1 & x<0,
\end{cases}
\end{align*}
on peut écrire 
\begin{align*}
sign(x) = \left(1 - \ind{x=0} \right) (-1)^{\ind{x>0} + \ind{x\neq 0}}
\end{align*}
\end{SousQuestion}


\begin{SousQuestion} % 1.6
\par Puisque $abs(x)$ est défini comme 
\begin{align*}
abs(x)=\abs{x}=\begin{cases}
x & x>0, \\
0  & x=0, \\
-x & x<0.
\end{cases}
\end{align*}
Puisque 
\begin{align*}
x > 0 & \implies \frac{d}{dx}abs(x)=1\\
x < 0 & \implies \frac{d}{dx}abs(x)=-1 \\
x = 0 & \implies \frac{d}{dx}abs(x) \myeq 0 
\end{align*}, on peut voir le parallèle avec la fonction $sign(x)$ défini précédemment. On obtient donc que 
\begin{align*}
abs(x)=sign(x)
\end{align*}
\end{SousQuestion}


\begin{SousQuestion} % 1.7
\par Puisque $rect(x)$ est défini comme 
\begin{equation*}
rect(x)=\left\{
\begin{split}
x & \quad x \geq 0, \\
0 & \quad x<0.
\end{split}
\right\} =[x]_{+}=max(0,x)= x \left(\ind{x>0}\right)
\end{equation*}
Puisque 
\begin{align*}
x > 0 & \implies \frac{d}{dx}rect(x)=1\\
x \leq 0 & \implies \frac{d}{dx}rect(x) \myeq 0,
\end{align*} on obtient l'écriture suivante 
\begin{align*}
\frac{d}{dx}rect(x)=\ind{x>0}
\end{align*}
\end{SousQuestion}


\begin{SousQuestion} % 1.8
\par On a d'abord que la norme $L_2$ est définie comme
\begin{align*}
L_2 = \norm{x}_2^2 = \somme{i}{}{x_i^2}.
\end{align*}
Puisque la dérivé d'une somme est la somme des dérivés et que $\frac{d}{dx_i}x_j^2 = 0, \ \forall i \neq j$, on a que le vecteur de dérivés partielle de la norme $L_2$ est
\begin{align*}
\frac{d}{dx} L_2 & = \left( \frac{d}{dx_1}\left(\somme{i}{}{x_i^2}\right), \frac{d}{dx_2}\left(\somme{i}{}{x_i^2}\right) , \dots \right) \\
& = \left( \left(\somme{i}{}{\frac{d}{dx_1}x_i^2}\right), \left(\somme{i}{}{\frac{d}{dx_2}x_i^2}\right) , \dots \right) \\
& = \left( \left(\frac{d}{dx_1}x_1^2 \right), \left(\frac{d}{dx_2}x_2^2\right) , \dots \right) \\
& = \left( 2x_1 , 2x_2 , \dots \right) \\
\end{align*}
\end{SousQuestion}


\begin{SousQuestion} % 1.9
\par On a d'abord que la norme $L_1$ est définie comme
\begin{align*}
L_1 = \norm{x}_1 = \somme{i}{}{\abs{x_i}}.
\end{align*}
Pour les mêmes raisons qu'énoncé ci-haut, on a que
\begin{align*}
\frac{d}{dx} L_1 & = \left( \frac{d}{dx_1}\left(\somme{i}{}{\abs{x_i}}\right), \frac{d}{dx_2}\left(\somme{i}{}{\abs{x_i}}\right) , \dots \right) \\
& = \left( \left(\somme{i}{}{\frac{d}{dx_1}\abs{x_i}}\right), \left(\somme{i}{}{\frac{d}{dx_2}\abs{x_i}}\right) , \dots \right) \\
& = \left( \left(\frac{d}{dx_1}\abs{x_1} \right), \left(\frac{d}{dx_2}\abs{x_2}\right) , \dots \right) \\
& = \left( sign(x_1) , sign(x_2) , \dots \right) \\
\end{align*}
\end{SousQuestion}

\end{Question}


\end{document}








